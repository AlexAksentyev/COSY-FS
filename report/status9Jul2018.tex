\documentclass{article}
\usepackage[utf8]{inputenc}
\usepackage[T1]{fontenc}
\usepackage[english, russian]{babel}
\usepackage{xfrac}
\usepackage{xparse}
\usepackage{amsfonts}
\usepackage{paralist}
\usepackage{subcaption}

\newcommand{\N}{\mathbb{N}}
\newcommand{\Wmdm}[1]{\Omega_{#1}^{MDM}}
\newcommand{\avg}[1]{\langle #1 \rangle}
\DeclareDocumentCommand{\bkt}{sm}{\IfBooleanTF{#1}{\left[ #2 \right]}{\left(#2\right)}}


\begin{document}
\section{Задачи}
\begin{enumerate}
\item Квантифицировать зависимость поперечной частоты прецессии спина $\Wmdm{x}$ от среднего угла наклона элементов ускорителя $\avg{\theta}$ (связь среднего значения угла со стандартным отклонением $\sigma_\theta$ распределения определяется статистическими выражениями); это график функции $(\avg{\theta}, \Wmdm{x})$.
\item Квантифицировать зависимость угла $\tan\alpha \equiv \sfrac{\Wmdm{x}}{\Wmdm{y}}$ между компонентами $\Wmdm{x},~\Wmdm{y}$ (для прямой CW и обратной CCW структуры) от:
  \begin{enumerate}
  \item точки инжекции в фазовом пространстве; функцию $((x,a,y,b,t,d)\vert_{s=0}, \alpha)$,
  \item от среднего угла $\avg{\theta}$: $(\avg{\theta}, \alpha)$,
  \end{enumerate}
\end{enumerate}

\section{Реализация}
\begin{enumerate}
\item Провести серию трекинговых симуляций ансамбля частиц $X\in \bkt*{-x_0, +x_0}, ~Y\in\bkt*{-y_0, +y_0}, ~D\in\bkt*{-d_0, +d_0}, \dots$ с семейством распределений
  \begin{enumerate}
  \item $\theta \sim N(\mu_0\cdot i, 0), i\in \N$, которая и даёт нам график $(\mu_i\equiv\avg{\theta}, \Wmdm{x})$, а также отдельно серию
  \item $\theta \sim N(0, \sigma_0\cdot j), j \in \N$, которая будет задавать границы точности юстировки, определяемые устойчивостью динамики пучка.
    \end{enumerate}
\item Для каждого рана, профитировать (либо использовать любой другой метод анализа сигнала) результаты трекинга спина $S_x, S_Y$ синусоидальным сигналом, для определения частот прецессии $\Wmdm{x}, ~\Wmdm{y}$, из которых тривиально получается $\tan\alpha$.
\item Повторить для CCW структуры.
  \item Повторить для другой структуры.
\end{enumerate}

\section{Что есть}
\begin{enumerate}
\item Обе структуры; все элементы обладают свойством поворачиваться вокруг оптической оси, при этом Вин-фильтры в BNL поворачиваются с сохранением (если я не ошибаюсь) вертикальной компоненты магнитного поля, и соответствующей подстройкой электрического.\footnote{Я просто домножаю оба поля $E,B = E_0,B_0/cos\theta$. По картинке, это должно быть правильно.}
\item В принципе, .fox скрипт теста, который трекает нужный ансамбль и пишет данные тоже есть.
\item Базовый .py скрипт для анализа и визуализации данных тоже есть. Пока я только фитировал данные, другими способами не определял частоты.
  \item Ну там, makefile вчера вечером написал, чтобы можно было удобно добавлять и компилировать тесты.
\end{enumerate}

\section{Какие проблемы}
\begin{enumerate}
\item $S_y$ никак не хочет расти до 1. Ниже представлены графики для распределения $N(0, 10^{-4}~ rad)$.
\item И частоты там плохие; 1 Гц видел максимум (на этих графиках в X-пучке видно), а так всё сильно меньше. Но на фоне того, что $S_y$ не растёт, эта проблема не кажется такой уж проблематичной; всё равно очевидно частота не этих колебаний нас интересует.
  \item Не совсем проблема, но я обнаружил что xz коррекция спина должна проводиться по углу отклонения какого-то одного луча (я использую квази-референсный, отклонённый по x на $10^{-6}$ мм), а не среднего. При достижении декогеренции в $\pi$ средний угол становится негладкой функцией, и получается что на предыдущем обороте коррекция $10^{-3}$, а в следующем 0.2. Через несколько таких коррекций сигнал портился и уже не похож был на синусоиду. Я попробовал поменять средний угол на медианный --- медиана более устойчивая характеристика среднего для косых распределений, --- не очень помогло. Помогает только разделение X-,Y-,D-пучков на отдельные ансамбли (там углы пооднороднее), ну и увеличение числа лучей в пучке. Но тоже не идеально. Идеально если корректировать по отклонению одного конкретного луча.
\end{enumerate}

\begin{figure}[ht]
  \centering
  \begin{subfigure}{\textwidth}
    \centering
    \includegraphics[scale=.8]{Sy_X_bunch}
    \caption{X-распределённый пучок}
  \end{subfigure}
  \begin{subfigure}{\textwidth}
    \centering
    \includegraphics[scale=.8]{Sy_Y_bunch}
    \caption{Y-распределённый пучок}
  \end{subfigure}
\end{figure}
\begin{figure}[ht]\ContinuedFloat
  \centering
  \begin{subfigure}{\textwidth}
    \centering
    \includegraphics[scale=.8]{Sy_D_bunch}
    \caption{D-распределённый пучок}
  \end{subfigure}
\end{figure}
\end{document}
